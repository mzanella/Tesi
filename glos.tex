\newglossaryentry{E-learning}{name=E-learning,description={Utilizzo di tecnologie multimediali e di Internet per migliorare la qualità dell'apprendimento facilitando l'accesso alle risorse e ai servizi, così come anche agli scambi in remoto e alla collaborazione}}
\newglossaryentry{xAPI}{name=xAPI,description={Vedi Experience API}}
\newglossaryentry{Tin Can API}{name=Tin Can API,description={Vedi Experience API}}
\newglossaryentry{Experience API}{name=Experience API,description={Specifica software per l'e-learning che definisce il modo in cui i contenuti e il sistema che li eroga devono dialogare. Le esperienze di apprendimento sono registrate in un Learning Record Store che può essere interno o esterno ad un Learning Management System}}
\newglossaryentry{SCORM}{name=SCORM,description={Acronimo di Shareable Content Object Reference Model, è una raccolta di specifiche tecniche che consente, primariamente, lo scambio di contenuti digitali in maniera indipendente dalla piattaforma. Nell'e-learning definisce le specifiche relative al riutilizzo, tracciamento e catalogazione degli oggetti didattici, strutturati in corsi}}
\newglossaryentry{LMS}{name=LMS, description={Acronimo di Learning Management System}}
\newglossaryentry{LRS}{name=LRS, description={Acronimo di Learning Record Store}}
\newglossaryentry{Learning Management System}{{name=Learning Management System}, description={Piattaforma applicativa (o insieme di programmi) che permette l'erogazione dei corsi in modalità e-learning al fine di contribuire a realizzare le finalità previste dal progetto educativo dell'istituzione proponente. Il learning management system presidia la distribuzione dei corsi on-line, l'iscrizione degli studenti e il tracciamento delle attività on-line}}
\newglossaryentry{Learning Record Store}{{name=Learning Record Store}, description={Repository utilizzato per salvare in modo permanente le esperienze di fruizione di corsi di e-learning che seguono la specifica Experience API}}
\newglossaryentry{JSON}{name=JSON, description={Acronimo di JavaScript Object Notation, formato adatto all'interscambio di dati fra applicazioni client-server. È basato sul linguaggio JavaScript Standard ECMA-262 3ª edizione dicembre 1999, ma ne è indipendente}}
\newglossaryentry{Statement}{name=Statement, description={Nel contesto dell'e-learning, dichiarazione utilizzata per il tracciamento di una attività di apprendimento}}
\newglossaryentry{Design pattern}{name=Design pattern, description={Soluzione progettuale generale ad un problema ricorrente}}
\newglossaryentry{Dependency injection}{name=Dependency injection, description={Design pattern il cui scopo è quello di semplificare lo sviluppo e migliorare la testabilità del software, diminuendo le dipendenze tra le varie componenti che lo compongono}}
\newglossaryentry{URI}{name=URI, description={Acronimo di Internationalized Resource Identifier, stringa che identifica univocamente una risorsa generica che può essere un indirizzo Web, un documento, un’immagine, un file, un servizio, ecc. e la rende disponibile tramite protocolli come HTTP o FTP}}
\newglossaryentry{IRI}{name=IRI, description={Acronimo di Internationalized Resource Identifier, forma generale di Uniform Resource Identifier costituita, a differenza di una URI, da una sequenza di caratteri appartenenti all'Universal Character Set (Unicode/ISO 10646), e ciò significa che al suo interno possono occorrere caratteri non appartenenti all'insieme ASCII}}
\newglossaryentry{Annotazione}{name=Annotazione, description={Modo per aggiungere metadati nel codice sorgente Java che possono essere disponibili al programmatore durante l'esecuzione}}
\newglossaryentry{Tap}{name=Tap, description={Breve tocco sullo schermo, solitamente utilizzato per la selezione di un elemento}}
