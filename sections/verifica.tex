\documentclass[../Tesi.tex]{subfiles}


\begin{document}
\section{Verifica e validazione}
	Per eseguire la verifica e la validazione del prodotto sono stati utilizzati il framework JUnit e Espresso. Il primo è un framework per atto allo sviluppo di unit test in Java, il secondo invece è una libreria per la creazione di test dell'interfaccia grafica in Android. \\
	l'implementazione dei test hai sia accompagnato che seguito lo sviluppo dell'applicazione. La maggior parte dei test di unità sono stati creati seguendo la metodologia test-driven development, che prevede che i test vengano scritti prima del codice che testano, in modo tale che ne guidino lo sviluppo. \\
	I test di unità che sono stati sviluppati coprono quasi la totalità dei metodi riguardanti le funzionalità principali dell'applicazione, mentre i test di integrazione e di sistema sono stati sviluppati in un numero inferiore, per mancanza di tempo. Per verificare il corretto funzionamento dell'applicazione è stato fatto un collaudo del prodotto insieme al tutor aziendale.

\end{document}