\documentclass[../Tesi.tex]{subfiles}


\begin{document}
\section{Verifica e validazione}
	Per eseguire la verifica e la validazione del prodotto sono stati utilizzati il framework JUnit e Espresso. Il primo strumento è un framework per lo sviluppo di unit test in Java, il secondo invece è una libreria per 
	simulare l'interazione di un utente con l'interfaccia grafica di applicazioni Android. \\
	L'implementazione dei test ha sia accompagnato che seguito lo sviluppo dell'applicazione. La maggior parte dei test di unità sono stati creati seguendo la metodologia test-driven development. Tale strategia prevede che la stesura dei test automatici avvenga prima di quella del software che deve essere sottoposto a test, e che lo sviluppo del software applicativo sia orientato all'obiettivo di passare i test automatici precedentemente predisposti. Più in dettaglio, la metodologia test-driven development prevede la ripetizione di un breve ciclo di sviluppo in tre fasi:
	\begin{enumerate}
		\item il programmatore scrive un test automatico per la nuova funzione da sviluppare, che deve fallire in quanto la funzione non è stata ancora realizzata;
		\item il programmatore sviluppa la quantità minima di codice necessaria per passare il test;
		\item il programmatore esegue il refactoring del codice per adeguarlo a determinati standard di qualità.
	\end{enumerate}
	I test di unità sviluppati coprono quasi la totalità dei metodi riguardanti le funzionalità principali dell'applicazione, mentre i test di integrazione e di sistema sono stati sviluppati in un numero inferiore, per mancanza di tempo. Per verificare il corretto funzionamento dell'applicazione è stato fatto un collaudo del prodotto insieme al tutor aziendale. Durante tale incontro sono stati inoltre discussi aspetti positivi, negativi e possibili sviluppi futuri dell'applicazione.\\
	Tutti i test effettuati sono stati svolti su di un dispositivo Motorola Moto G 2014 con sistema operativo Android 5.2.

	\subsection{Requisiti soddisfatti}
	In base agli obiettivi prefissati dello stage, definiti nella sezione \textit{Obiettivi}, si possono trarre le seguenti conclusioni:
	\begin{itemize}
		\item gli obiettivi obbligatori, previsti dal piano di lavoro, sono stati tutti soddisfatti;
		\item gli obiettivi opzionali sono stati parzialmente soddisfatti.
	\end{itemize}
	In particolare è parzialmente soddisfatto il requisito OP1.1: l'applicazione non prevede, infatti, la possibilità di cambiare i font utilizzati. Tale funzionalità non è stata implementata per mancanza di tempo e perché è stata data priorità ai requisiti che potessero dare un maggior valore aggiunto all'applicazione.
\end{document}