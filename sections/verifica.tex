\documentclass[../Tesi.tex]{subfiles}


\begin{document}
\section{Verifica e validazione}
	Per eseguire la verifica e la validazione del prodotto sono stati utilizzati il framework JUnit e Espresso. Il primo strumento è un framework per lo sviluppo di unit test in Java, il secondo invece è una libreria per 
	simulare l'interazione di un utente con l'interfaccia grafica di applicazioni Android. \\
	L'implementazione dei test ha sia accompagnato che seguito lo sviluppo dell'applicazione. La maggior parte dei test di unità sono stati creati seguendo la metodologia test-driven development. Tale strategia prevede che la codifica dei test preceda la scrittura del codice per i quali sono stati pensati, in modo tale che ne guidino lo sviluppo. \\
	I test di unità sviluppati coprono quasi la totalità dei metodi riguardanti le funzionalità principali dell'applicazione, mentre i test di integrazione e di sistema sono stati sviluppati in un numero inferiore, per mancanza di tempo. Per verificare il corretto funzionamento dell'applicazione è stato fatto un collaudo del prodotto insieme al tutor aziendale. Durante tale incontro sono stati inoltre discussi aspetti positivi, negativi e possibili sviluppi futuri dell'applicazione.

\end{document}