\documentclass[../Tesi.tex]{subfiles}


\begin{document}
\section{Verifica e validazione}\label{sec:VerificaEValidazione}
	Per eseguire la verifica e la validazione del prodotto sono stati utilizzati il framework JUnit e Espresso. Il primo strumento è un framework per lo sviluppo di unit test in Java, il secondo, invece, è una libreria per 
	simulare l'interazione di un utente con l'interfaccia grafica di applicazioni Android. \\
	I test sviluppati sono di tre tipologie:
	\begin{itemize}
		\item test di unità;
		\item test di integrazione;
		\item test di sistema.
	\end{itemize}
	L'implementazione dei test ha sia accompagnato che seguito lo sviluppo dell'applicazione. La maggior parte dei test di unità sono stati creati seguendo la metodologia test-driven development. Tale strategia prevede che la stesura dei test automatici avvenga prima di quella del software che deve essere sottoposto a test, e che lo sviluppo del codice sia orientato all'obiettivo di passare i test automatici precedentemente predisposti. Più in dettaglio, la metodologia test-driven development prevede la ripetizione di un breve ciclo di sviluppo in tre fasi:
	\begin{enumerate}
		\item il programmatore scrive un test automatico per la nuova funzione da sviluppare, che deve fallire in quanto la funzione non è stata ancora realizzata;
		\item il programmatore sviluppa la quantità minima di codice necessaria per passare il test;
		\item il programmatore esegue il refactoring del codice per adeguarlo a determinati standard di qualità.
	\end{enumerate}
	I test di unità sviluppati hanno permesso da subito di evidenziare la maggior parte degli errori presenti nel codice. Tale tipologia di test è stata sviluppata per ogni classe in modo da coprire la quasi totalità dei metodi. I test di integrazione e di sistema sviluppati, invece, sono in numero ridotto e sono stati sviluppati solamente per alcune funzionalità centrali del prodotto. Il motivo di un numero limitato di test risiede nel poco tempo disponibile per terminare l'applicazione. I test di integrazione che sono stati sviluppati riguardano principalmente le classi che si occupano dell'autenticazione e registrazione di un utente, della persistenza dei dati relativi ai corsi che devono essere gestiti e della memorizzazione degli statement. I test di sistema, invece, sono stati sviluppati al fine di simulare l'interazione di un utente con le funzionalità principali dell'applicazione.\\
	Al fine di verificare il corretto funzionamento di quanto è stato sviluppato, negli ultimi giorni di stage è stato organizzato un collaudo del prodotto al quale ha partecipato anche il tutor aziendale. Tale incontro è stato utile per evidenziare gli aspetti positivi e i punti deboli dell'applicazione, oltre che per sistemare alcuni aspetti grafici.\\
	Tutti i test, come il collaudo, sono stati effettuati su di un dispositivo Motorola Moto G 2014 con sistema operativo Android 5.2.

	\subsection{Risultati raggiunti}
	In seguito ai test sviluppati ed al collaudo realizzato, facendo riferimento agli obiettivi prefissati dello stage (riportati nella sezione \hyperref[subsec:obiettivi]{\textit{Obiettivi}}), è possibile trarre le seguenti conclusioni:
	\begin{itemize}
		\item gli obiettivi obbligatori, previsti dal piano di lavoro, sono stati tutti soddisfatti;
		\item gli obiettivi opzionali sono stati parzialmente soddisfatti.
	\end{itemize}
	In particolare è parzialmente soddisfatto il requisito \hyperref[op1.1]{OP1.1}: l'applicazione non prevede, infatti, la possibilità di cambiare i font utilizzati. Tale funzionalità non è stata implementata per mancanza di tempo e perché è stata data priorità ai requisiti che potessero dare un maggior valore aggiunto all'applicazione.
\end{document}