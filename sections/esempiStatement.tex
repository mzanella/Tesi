\section{Esempi di statement} \label{App:AppendixA}
    \subsection{Statement di inizio di un contenuto}
    Di seguito troviamo l'esempio di uno statement che viene generato quando un utente inizia un contenuto. Tale statement può essere letto come \textit{``Marco Zanella ha iniziato il corso Fondo Casse e Spese Minute il 23 luglio 2016 alle 18:30''}.
\begin{lstlisting}
{
    "version": "1.0.0",
    "actor": {
        "objectType": "Agent",
        "name": "Marco Zanella",
        "mbox": "mailto:marco@email.com"
    },
    "verb": {
        "id": "http://adlnet.gov/expapi/verbs/initialized",
        "display": {
            "en-US": "initialized"
        }
    },
    "object": {
        "objectType": "Activity",
        "id": "http://Fondo%20Cassa%20e%20Spese%20Minute",
        "definition": {
            "description": {
                "und": "Fondo Cassa e Spese Minute"
            },
            "type": "http://adlnet.gov/expapi/activities/module"
        }
    },
    "context": {
        "contextActivities": {
            "parent": [
                {
                    "objectType": "Activity",
                    "id": "http://Fondo%20Cassa%20e%20Spese%20Minute"
                }
            ],
            "grouping": [
                {
                    "objectType": "Activity",
                    "id": "http://Fondo%20Cassa%20e%20Spese%20Minute"
                }
            ]
        }
    },
    "id": "cef4834e-6dc5-4c01-8979-e0222dc8a100",
    "authority": {
        "objectType": "Agent",
        "name": "New Client",
        "mbox": "mailto:hello@learninglocker.net"
    },
    "stored": "2016-07-23T18:30:06.791800+02:00",
    "timestamp": "2016-07-23T18:30:06.791800+02:00"
}
\end{lstlisting}

	\subsection{Statement di visualizzazione di una slide di un contenuto}
    Di seguito troviamo l'esempio di uno statement che viene generato quando un utente visualizza una slide che non prevede un test. Tale statement può essere letto come \textit{``Marco Zanella ha fatto pratica sulla slide Miroglio Fashion del corso Fondo Casse e Spese Minute il 23 luglio 2016 alle 18:30''}.
\begin{lstlisting}
{
    "version": "1.0.0",
    "actor": {
        "objectType": "Agent",
        "name": "Marco Zanella",
        "mbox": "mailto:marco@email.com"
    },
    "verb": {
        "id": "http://adlnet.gov/expapi/verbs/experienced",
        "display": {
            "en-US": "experienced"
        }
    },
    "object": {
        "objectType": "Activity",
        "id": "http://Fondo%20Cassa%20e%20Spese%20Minute/6KswNmNiPKX",
        "definition": {
            "name": {
                "en-US": "Miroglio Fashion"
            },
            "description": {
                "en-US": "Fondo Cassa e Spese Minute"
            },
            "type": "http://adlnet.gov/expapi/activities/module"
        }
    },
    "context": {
        "contextActivities": {
            "parent": [
                {
                    "objectType": "Activity",
                    "id": "http://Fondo%20Cassa%20e%20Spese%20Minute"
                }
            ],
            "grouping": [
                {
                    "objectType": "Activity",
                    "id": "http://Fondo%20Cassa%20e%20Spese%20Minute"
                }
            ]
        }
    },
    "id": "5f393687-79ca-4f0b-bcd9-7014fa37a984",
    "authority": {
        "objectType": "Agent",
        "name": "New Client",
        "mbox": "mailto:hello@learninglocker.net"
    },
    "stored": "2016-07-23T18:30:07.820700+02:00",
    "timestamp": "2016-07-23T18:30:07.820700+02:00"
}
\end{lstlisting}
    
    \subsection{Statement di risposta ad una domanda di un test}
    Di seguito troviamo l'esempio di uno statement che viene generato quando un utente risponde ad una slide che prevede un test. Tale statement può essere letto come \textit{``Marco Zanella ha risposto alla domanda Questo è un vero o falso del corso DEMO\_TEST il 23 luglio 2016 alle 11:26''}.
\begin{lstlisting}
{
    "version": "1.0.0",
    "actor": {
        "objectType": "Agent",
        "name": "Marco Zanella",
        "mbox": "mailto:marco@email.com"
    },
    "verb": {
        "id": "http://adlnet.gov/expapi/verbs/answered",
        "display": {
            "en-US": "answered"
        }
    },
    "object": {
        "objectType": "Activity",
        "id": "http://6PdqP7rkRtE_course_id/618Dl3BgPcO",
        "definition": {
            "name": {
                "en-US": "Questo \u00e8 un vero o falso %Variable1%"
            },
            "description": {
                "en-US": "DEMO_TEST"
            },
            "type": "http://adlnet.gov/expapi/activities/module"
        }
    },
    "result": {
        "score": {
            "scaled": 0.3333,
            "raw": 10,
            "max": 10
        },
        "success": true,
        "response": "true"
    },
    "context": {
        "contextActivities": {
            "parent": [
                {
                    "objectType": "Activity",
                    "id": "http://6PdqP7rkRtE_course_id"
                }
            ],
            "grouping": [
                {
                    "objectType": "Activity",
                    "id": "http://6PdqP7rkRtE_course_id"
                }
            ]
        }
    },
    "id": "c7f63cf1-54a4-45d3-b5e2-06e5d2561860",
    "authority": {
        "objectType": "Agent",
        "name": "New Client",
        "mbox": "mailto:hello@learninglocker.net"
    },
    "stored": "2016-07-23T11:26:29.352200+02:00",
    "timestamp": "2016-07-23T11:26:29.352200+02:00"
}
\end{lstlisting}

    \subsection{Statement finale di un contenuto che comprende dei test}
    Di seguito troviamo l'esempio di uno statement che viene generato quando un utente termina un corso che comprende un test. Tale statement può essere letto come \textit{``Marco Zanella ha terminato il corso DEMO\_TEST, con il punteggio di 20/30, non superando il corso, il 23 luglio 2016 alle 11:26''}.
\begin{lstlisting}
{
    "version": "1.0.0",
    "actor": {
        "objectType": "Agent",
        "name": "Marco Zanella",
        "mbox": "mailto:marco@email.com"
    },
    "verb": {
        "id": "http://adlnet.gov/expapi/verbs/terminated",
        "display": {
            "en-US": "terminated"
        }
    },
    "object": {
        "objectType": "Activity",
        "id": "http://6PdqP7rkRtE_course_id",
        "definition": {
            "description": {
                "und": "DEMO TEST"
            },
            "type": "http://adlnet.gov/expapi/activities/module"
        }
    },
    "result": {
        "score": {
            "scaled": 0.6666,
            "raw": 20,
            "max": 30
        },
        "success": false
    },
    "context": {
        "contextActivities": {
            "parent": [
                {
                    "objectType": "Activity",
                    "id": "http://6PdqP7rkRtE_course_id"
                }
            ],
            "grouping": [
                {
                    "objectType": "Activity",
                    "id": "http://6PdqP7rkRtE_course_id"
                }
            ]
        }
    },
    "id": "f687a812-587d-4159-bbd8-b53f44ed34ca",
    "authority": {
        "objectType": "Agent",
        "name": "New Client",
        "mbox": "mailto:hello@learninglocker.net"
    },
    "stored": "2016-07-23T11:26:38.193700+02:00",
    "timestamp": "2016-07-23T11:26:38.193700+02:00"
}
\end{lstlisting}