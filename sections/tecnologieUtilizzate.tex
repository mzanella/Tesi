\documentclass[../Tesi.tex]{subfiles}


\begin{document}
\section{Tecnologie utilizzate}
	
	\subsection{Android}
		Android è un sistema operativo mobile sviluppato da Google e basato su kernel Linux. È stato progettato per essere eseguito principalmente su smartphone e tablet, con interfacce utente specializzate per orologi e televisori. Le versioni supportate dall'applicazione sono la 4.2 e superiori.
		\subsubsection{Vantaggi}
			\begin{itemize}
				\item possiede una vasta fetta di mercato mobile;
				\item disponibile su un vasto numero di dispositivi;
				\item quasi totalmente gratuito ed open-source.
			\end{itemize}
		\subsubsection{Svantaggi}
			\begin{itemize}
				\item essendoci un vasto numero di produttori di smartphone e tablet che non aggiornano la versione di Android che rilasciano all'interno dei loro dispositivi, Android risulta essere estremamente frammentato.
			\end{itemize}
			
	\subsection{Java}
		Java è uno dei più famosi linguaggi di programmazione orientato agli oggetti supportato da una moltitudine di librerie e documentazione. Viene utilizzato per la scrittura e lo sviluppo dell'applicazione.
		\subsubsection{Vantaggi}
			\begin{itemize}
				\item linguaggio più conosciuto, diffuso e utilizzato nell'ambiente di sviluppo Android
				\item ampia documentazione disponibile;
				\item dispone di un gran numero di librerie;
				\item portabilità su diversi sistemi operativi.
			\end{itemize}
		\subsubsection{Svantaggi}
			\begin{itemize}
				\item linguaggio verboso.
			\end{itemize}

	\subsection{JavaScript}
		JavaScript è un un linguaggio di scripting orientato agli oggetti e agli eventi, comunemente utilizzato nella programmazione Web lato client per la creazione, in siti web e applicazioni web, di effetti dinamici interattivi tramite funzioni di script invocate da eventi innescati a loro volta in vari modi dall'utente sulla pagina web in uso. L'utilizzo di tale tecnologia si è rilevata necessaria per recuperare le informazioni di fruizione dei contenuti di un utente e mandarle all'applicazione Android. 
		\subsubsection{Vantaggi}
			\begin{itemize}
				\item permette l'interazione con pagine web e con i contenuti xAPI;
				\item è eseguito lato client.
			\end{itemize}
		\subsubsection{Svantaggi}
			\begin{itemize}
				\item problemi di sicurezza, soprattutto in ambiente Android.
			\end{itemize}

	\subsection{MPandroidChart}
		MPAndroidChart è una libreria per la creazione di grafici in Android. Essa viene usata per la creazione di grafici a torta nell'applicazione, ma ne supporta molti altri. Tale libreria fornisce anche delle funzioni di ridimensionamento, trascinamento, selezione e di animazione. Tale scelta è stata fatta principalmente per le poche alternative disponibili.
		\subsubsection{Vantaggi}
			\begin{itemize}
				\item creazione semplice di grafici;
				\item grafici ampiamente personalizzabili;
				\item legende create automaticamente e personalizzabili.
			\end{itemize}
		\subsubsection{Svantaggi}
			Non sono stati riscontrati particolari svantaggi nell'utilizzo di tale libreria, poichè accompagnata da una buona documentazione.

	\subsection{Picasso}
		Picasso è una libreria Android utile per download, caching e mostrare le immagini. Tale libreria è stata utilizzata per la gestione di tutte le immagini relative ai contenuti presenti nell'applicazione.
		\subsection{Vantaggi}
			\begin{itemize}
				\item utilizzo molto semplice;
				\item permette di gestire immagini nella memoria del dispositivo o online indifferentemente.
			\end{itemize}
		\subsection{Svantaggi}
			\begin{itemize}
				\item qualche problema nel ridimensionamento delle immagini per inserirle nelle ImageView di Android.
			\end{itemize}

	\subsection{TinCanAndroid-Offline}
		TinCanAndroid-Offline è una libreria utilizzata per la creazione, il salvataggio in locale e l'invio di statement in formato xAPI. È stata adottata poichè è l'unica libreria open-source che fornisce tale funzionalità e non sono state trovate alternative valide nemmeno tra le librerie non disponibili gratuitamente.
		\subsubsection{Vantaggi}
			\begin{itemize}
				\item open-source;
				\item non sono state trovate delle valide alternative;
				\item è possibile accedere al codice sorgente su Github;
				\item gestisce in maniera abbastanza semplice il salvataggio in locale e l'invio degli statement.
			\end{itemize}
		\subsubsection{Svantaggi}
			\begin{itemize}
				\item documentazione praticamente assente;
				\item pochi esempi di utilizzo disponibili;
				\item non è disponibile nei repository dove vengono pubblicate le librerie di cui è possibile semplicemente dichiarare la dipendenza nel file Gradle per la compilazione di un progetto Android Studio;
				\item non è disponibile un file JAR da includere in un progetto Android;
				\item non sempre intuitiva nell'utilizzo.
			\end{itemize}

	\subsection{Google AutoValue}
		È una libreria che permetta la generazione automatica del codice delle classi annotate tramite delle annotazioni proprie della libreria. Tale libreria permette di creare gli oggetti di tale classi utilizzando il design pattern Builder. È utilizzata per le classi che richiedono, nel loro costruttore, potrebbero richiedere molti parametri, evitando così il fenomeno del telescoping. 
		\subsubsection{Vantaggi}
			\begin{itemize}	
				\item evita al programmatore la creazione di classi semplici ma ripetitive, diminuendo la possibilità di errori;
				\item mette a disposizione il pattern Builder per la creazione di oggetti;
				\item le classi vengono generate in fase di compilazione.
			\end{itemize}
		\subsubsection{Svantaggi}
			\begin{itemize}
				\item le classi create non sono disponibili tra quelle visibili nel sorgente;
				\item non è sempre riconosciuto subito dagli IDE, che possono scambiare l'utilizzo di tale libreria per errori.
			\end{itemize}

	\subsection{Gson}
		È una libreria per la gestione di oggetti in formato Json in Android. Tale libreria è utilizzata sia per la lettura dei contenuti disponibili pubblicati sul server, sia per la trasformazione degli oggetti scambiati tra Java e JavaScript. È stata scelta tale libreria per la semplicità si utilizzo.
		\subsubsection{Vantaggi}
			\begin{itemize}
				\item molto intuitiva;
				\item permette la trasformazione di oggetti Java in oggetti in formato Json e viceversa.
			\end{itemize}	
		\subsubsection{Svantaggi}
			\begin{itemize}
				\item più lenta di altre librerie che offrono le stesse funzionalità.
			\end{itemize}

	\subsection{Dagger 2}
		È una libreria che permette la dependency injection in Android. Nel progetto è largamente utilizzato tale design pattern per diminuire il più possibile le dipendenze e ciò permette all'applicazione di essere facilmente sviluppata dopo lo stage. La scelta di tale libreria è stata fatta poichè permette di effettuare l'injection di una qualsiasi classe, a differenza di altre di librerie come RoboGuice o Butterknife.
		\subsubsection{Vantaggi}
			\begin{itemize}
				\item non rallenta l'applicazione a runtime;
				\item le risoluzioni delle dipendenze possono essere dichiarate come dei comuni metodi, sfruttando le appropriate annotazioni.
			\end{itemize}
		\subsubsection{Svantaggi}
			\begin{itemize}
				\item non è sempre riconosciuto subito dagli IDE, che possono scambiare l'utilizzo di tale libreria per errori;
				\item i campi che devono essere inizializzati tramite dependency injection non possono avere visibilità privata o protetta.
			\end{itemize}

	\subsection{Realm}
		Realm è una libreria che offre le funzionalità di utilizzo di un database con le operazioni CRUD e la trasformazione di oggetti del database in oggetti Java in modo automatico.
		\subsubsection{Vantaggi}
			\begin{itemize}
				\item semplicità di utilizzo;
				\item accesso veloce ai dati;
				\item thread-safe;
				\item trasformazione automatica tra oggetti del database e oggetti Java.
			\end{itemize}
		\subsubsection{Svantaggi}
			\begin{itemize}
				\item non è intuitivo il comportamento nel multithreading;
				\item tutti gli oggetti che devono essere salvati nel database devono estendere una specifica classe.
			\end{itemize}
\end{document}