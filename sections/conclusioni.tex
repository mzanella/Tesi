\documentclass[../Tesi.tex]{subfiles}


\begin{document}
\section{Conclusioni}
	\subsection{Riassunto del lavoro svolto}
	La prima parte dello stage si è concentrata sullo studio del progetto e delle tecnologie richiesta da quest'ultimo. Il punto di partenza è stato, infatti, imparare come funziona la specifica xAPI, come creare degli statement che rispettino tale specifica e, sopratutto, come ricavare i dati di interazione di un utente con un corso per poi costruire gli statement. Questa fase ha richiesto di analizzare attentamente la libreria TinCanAndroid-Offline e si è evidenziata la necessità di creare uno script JavaScript che potesse in qualche modo passare le informazioni di fruizione di un utente all'applicazione Android. Questa parte di stage ha richiesto, inoltre, la ricerca di un LRS a cui, una volta sviluppato il prodotto, inviare gli statement. Terminata questa fase, è iniziata la progettazione dell'applicazione. Questa parte di stage è stata accompagnata dalla creazione di una serie di prototipi, con i quali perfezionare le soluzioni progettuali pensate. Il risultato di questa fase è stata l'architettura di base dell'applicazione, comprensiva di tutti i requisiti obbligatori. Tale struttura è stata ritoccata durante le fasi successive per comprendere anche gli obiettivi opzionali. La fase di implementazione ha seguito la progettazione. In questa parte di stage sono state sviluppate sia le classi che compongono l'applicazione, sia una buona parte di test. La fase conclusiva, quella di test, ha permesso di perfezionare l'applicazione, sia da un punto di vista grafico, sia delle funzionalità, grazie allo sviluppo di altri test.\\
	Il prodotto finale, anche se soddisfa la maggior parte degli obiettivi prefissati, presenta comunque alcune problematiche. Esse sono dovute principalmente alla mia scarsa esperienza, in particolar modo nel ruolo da progettista. Una delle problematiche può riguardare l'utilizzo del pattern MVP: è possibile infatti che alcune componenti della \textit{view} svolgano mansioni attribuibili al \textit{presenter}, o viceversa. Un altro problema relativo alla progettazione dell'applicazione riguarda il non aver fissato delle metriche di qualità da seguire nello sviluppo dal codice. Pur trattandosi di un caso isolato, ciò ha permesso alla classe \mclass{DataAccess}, che si occupa di esporre tutti i metodi per il recupero di informazioni dal database locale, di avere un numero piuttosto cospicuo di metodi e, di conseguenza, di statement. Negli sviluppi futuri potrebbe essere necessario ristrutturare tale classe, in modo da conferire maggiore manutenibilità al codice. Altre problematiche potrebbero riguardare l'usabilità dell'interfaccia grafica, che potrebbe non essere ottimale. Infine il prodotto dovrebbe essere testato ulteriormente. La fase di studio di fattibilità, infatti, ha richiesto qualche giorno in più di quanti erano stati preventivati nel piano di progetto e ciò ha inciso ancor di più sulla fase di test, che prevedeva, in ogni caso, un numero di giorni abbastanza limitato.

	\subsection{Sviluppi futuri}
	Alla fine dello stage, oltre a sistemare alcune problematiche evidenziate, esistono anche altre prospettive di miglioramento del prodotto realizzato:
	\begin{itemize}
		\item \textbf{Sviluppo di un applicazione iOS e/o Windows Phone per ampliare il bacino di utenza}: al momento il servizio è disponibile esclusivamente per dispositivi mobile con sistema operativo Android. Sarebbe meglio sviluppare applicazioni anche per gli altri sistemi operativi per dispositivi mobili, per poter ricoprire una fascia di mercato più ampia;
		\item \textbf{Miglioramento del tracciamento delle attività di un utente}: in questo momento l'applicazione traccia solamente le azioni di inizio e fine di un corso e i dati di fruizione delle singole slide. Potrebbe essere interessante ampliare tale funzionalità, permettendo di inviare all'LRS statement che comprendono quando un utente si è loggato, quando un utente si è registrato oppure altre azioni eseguite sull'applicazione;
		\item \textbf{Aggiungere la possibilità di registrarsi utilizzando i social}: questa funzionalità, anche se non vitale ai fini del prodotto, potrebbe velocizzare la fase di registrazione e autenticazione, dando la possibilità agli utenti, in versioni successive, di condividere i risultati ottenuti nei vari corsi;
		\item \textbf{Miglioramento del backend}: l'applicazione si appoggia su componenti esterne, le quali si occupano dell'autenticazione, registrazione e il recupero dei dati dei contenuti da visualizzare, che però non sono state sufficientemente sviluppate per mancanza di tempo. Tali componenti dovrebbero essere migliorate. Molto utile potrebbe essere, inoltre, lo sviluppo di un applicativo o di un sito web che, facendo delle query all'LRS, riuscisse a fornire delle informazioni che riguardino tutti gli utenti del prodotto. Ciò potrebbe essere molto utile per avere delle informazioni statistiche globali dell'applicazione per vedere, per esempio, quanti utenti hanno fruito di un determinato corso o quanti hanno superato una determinata domanda. Tale strumento permetterebbe infatti, oltre ad avere una panoramica sugli utenti senza andare ad interrogare con l'LRS con delle query a basso livello, di migliorare i corsi, avendo informazioni generali sulla loro fruizione.
	\end{itemize}

	\subsection{Considerazioni finali e conoscenze acquisite}
	Dal punto di vista formativo l'attività di stage è stata piuttosto positiva. Ho potuto infatti approfondire la conoscenza di varie tecnologie, in particolare riguardanti il framework Android. Oltre a ciò lo stage è stato utile poichè mi ha dato l'opportunità di accrescere la mia autonomia e la capacità di farmi carico di responsabilità. Infatti il progetto è stato portato avanti solamente da me, e, in più di un occasione nelle fasi di progettazione e sviluppo, ho avuto delle problematiche per le quali ho dovuto spendere diversi giorni per superarle. Inoltre, essendo tra le mie prime esperienze lavorative, è stato importante confrontarmi con il mondo del lavoro e mi ha permesso di capire cosa vuol dire lavorare all'interno di un azienda. Questo progetto, però, non mi ha dato l'opportunità di accrescere le mie capacità di lavorare all'interno di un team, né di lavorare assieme a persone con esperienza nella progettazione e sviluppo del software. Ciò avrebbe potuto conferire allo stage un'importanza ancora maggiore dal punto di vista formativo.\\
	In ogni caso, le conoscenze apprese in questi tre anni di università mi hanno aiutato sia a comprendere nuove tecnologie con facilità, sia ad affrontare i problemi con metodo e flessibilità, in modo da poter trovare una soluzione. 
\end{document}