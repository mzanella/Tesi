\documentclass[../Tesi.tex]{subfiles}

\begin{document}
\section{Progetto dello stage}
	\subsection{Motivazione}
	La specifica xAPI ha aperto una nuova serie di possibilità per la fruizione di contenuti per l'e-learnig. Infatti, non avendo più la necessità di avere una connessione Internet sempre attiva, è possibile creare delle applicazioni per dispositivi mobile: quest'ultime, dando la possibilità di effettuare il download dei contenuti, permettono agli utenti di fruirne in qualsiasi luogo e momento. Inoltre, le persone si stanno abituando sempre più all'utilizzo delle app, le quali riscontrano successo poichè riducono il tempo di accesso alle informazioni desiderate. Nel caso specifico un'applicazione può evitare ad un utente di dover aprire un browser, cercare su di un motore di ricerca la piattaforma LMS a cui collegarsi oppure scriverne l'URL, entrare nella piattaforma e, eventualmente, effettuare il login. In più, sfruttare la specifica xAPI può risultare conveniente anche per chi richiede contenuti per l’e-learning: infatti se si necessita solamente di permettere la fruizione di alcuni corsi non è necessario lo sviluppo di un LMS o il pagamento di una tale piattaforma ma è possibile sfruttare un qualunque spazio web per la pubblicazione dei contenuti. Per il recupero delle esperienze degli utenti, invece, è possibile utilizzare un LRS. Quest'ultimo, generalmente, non ha particolari necessità in termini di requisiti hardware della macchina su cui è installato e ne sono disponibili diversi open-source (per esempio LearningLocker). Infine è utile anche a chi eroga corsi di e-learning, poichè è possibile fornire ai clienti un’ulteriore strumento per la fruizione dei contenuti.

	\subsection{Descrizione del prodotto finale}
	Il prodotto finale consiste in un'applicazione per la fruizione e il tracciamento di contenuti xAPI sia online che offline. I contenuti che possono essere fruiti dall’applicazione sono specificati all'interno di un file in formato JSON che risiede in un server raggiungibile online. La locazione di tale file deve essere specificata in un file di configurazione dell'applicazione stessa. Il prodotto, inoltre, permette ad un utente l'autenticazione e la registrazione: ciò è fondamentale per poter associare ogni azione significativa, che dev'essere tracciata, alla persona che l'ha eseguita. Ad ogni avvio, in caso di connessione Internet attiva, l'applicazione controlla:
	\begin{itemize}
		\item se sono disponibili aggiornamenti ai contenuti disponibili: in tal caso viene aggiornato il database locale al dispositivo, in cui sono memorizzati i contenuti fruibili dall'applicazione;
		\item se sono state registrate nell'LRS azioni di un certo utente di cui non è presente traccia in locale: in tal caso vengono aggiornati i dati relativi alle esperienze dell'utente, anch'essi registrati nel database del dispositivo.
	\end{itemize} 
	Per ogni contenuto un utente po' scegliere se effettuarne il download oppure fruirne online.
	L'applicazione, inoltre, permette, ad un utente autenticato, di accedere ad un riepilogo delle attività riguardanti i contenuti di e-learning fruibili. 

	\subsection{Obiettivi}
		Gli obiettivi dello stage sono stati individuati assieme al tutor aziendale Germana Boghetto e al responsabile dell'area informatica Marco Petrin. Sono stati suddivisi in:
		\begin{itemize}
			\item obiettivi obbligatori: requisiti minimi che devono essere soddisfatti dall'applicazione alla fine dello stage;
			\item obiettivi opzionali.
		\end{itemize}

		\subsubsection{Obiettivi obbligatori}
			\begin{longtabu} to \textwidth {X[0.3cm] X}
				\toprule
				\textbf{Identificativo} & \textbf{Descrizione} \\
				\midrule
				\endhead
				\arrayrulecolor{gray}
				OB.1 & L’applicazione deve permettere la visualizzazione di oggetti didattici in formato xAPI \\
				\midrule
				OB.1.1 & L’applicazione deve permettere la fruizione degli oggetti didattici sia in modalità online che offline del dispositivo \\
				\midrule
				OB.2 & L’applicazione deve consentire l’interazione tra l’utente e l’oggetto didattico come da funzionalità implementate nell’oggetto didattico stesso \\
				\midrule
				OB.3 & L’applicazione deve tracciare i dati di fruizione dell’utente all’interno dell’oggetto didattico \\
				\midrule
				OB.3.1 & L’applicazione deve permettere di registrare i dati di più oggetti didattici differenti \\
				\midrule
				OB.3.2 & L’applicazione deve permettere di estrarre, inviare e/o visualizzare i report di fruizione degli utenti sui vari oggetti didattici \\
				\midrule
				OB.4 & L’applicazione deve funzionare su dispositivi Android \\
				\arrayrulecolor{black}
				\bottomrule
			\end{longtabu}

			\subsubsection{Obiettivi opzionali}
			\begin{longtabu} to \textwidth {X[0.3cm] X}
				\toprule
				\textbf{Identificativo} & \textbf{Descrizione} \\
				\midrule
				\endhead
				\arrayrulecolor{gray}
				OP.1 & L’applicazione può essere personalizzabile graficamente a seconda delle esigenze del cliente  \\
				\midrule
				OP.1.1 & La personalizzazione grafica può comprendere la modifica dei colori, di un eventuale logo e dei font presenti all’interno dell’applicazione  \\
				\midrule
				OP.2 & L’applicazione può permettere la profilazione di diversi utenti  \\
				\midrule
				OP.2.1 & La profilazione può essere determinata da una piattaforma che sta a monte dell’oggetto didattico  \\
				\midrule
				OP.2.2 & La profilazione con utenti diversi comporta diversi report all’interno dell’applicazione \\
				\arrayrulecolor{black}
				\bottomrule
			\end{longtabu}

	\subsection{Vincoli tecnologici}
		Non è stato imposto alcun vincolo particolare sulle tecnologie da utilizzare nello sviluppo dell'applicazione. Le uniche tecnologie di cui è richiesto l'utilizzo sono:
		\begin{itemize}
			\item Android;
			\item contenuti xAPI.
		\end{itemize}
\end{document}